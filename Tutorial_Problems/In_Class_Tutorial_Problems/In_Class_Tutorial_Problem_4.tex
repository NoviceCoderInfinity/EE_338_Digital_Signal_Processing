\documentclass{article}
\usepackage{helvet}
\usepackage{geometry}
\usepackage{graphicx}
\usepackage{amsmath}
\usepackage{hyperref}
\usepackage{xcolor}
\usepackage{titlesec}
\usepackage{microtype} % Prevents overfull hboxes by better text wrapping
\geometry{margin=1in}
\usepackage{subcaption}
\usepackage{listings}
\usepackage{color}
% Define colors for code syntax highlighting
\definecolor{codeblue}{rgb}{0.13, 0.13, 0.7}
\definecolor{codegreen}{rgb}{0, 0.5, 0}
\definecolor{codegray}{rgb}{0.5, 0.5, 0.5}
\definecolor{codepurple}{rgb}{0.58, 0, 0.82}

\lstdefinestyle{mystyle}{
    backgroundcolor=\color{white},   
    commentstyle=\color{codegreen},
    keywordstyle=\color{codeblue},
    numberstyle=\tiny\color{codegray},
    stringstyle=\color{codepurple},
    basicstyle=\ttfamily\footnotesize,
    breaklines=true,                 
    captionpos=b,                    
    keepspaces=true,                 
    numbers=left,                    
    numbersep=5pt,                  
    showspaces=false,                
    showstringspaces=false,
    showtabs=false,                  
    tabsize=2
}

\lstset{style=mystyle}

% Define colors
\definecolor{primary}{RGB}{0, 102, 204} % Blue color
\definecolor{IITBBlue}{RGB}{0, 51, 102} % IIT Bombay's signature blue

\usepackage{cite}
\usepackage{amsmath,amssymb,amsfonts}
\usepackage{algorithmic}
\usepackage{graphicx}
\usepackage{textcomp}
\usepackage{xcolor}
\usepackage{hyperref}
\usepackage{placeins}
\usepackage{graphicx}
\usepackage{subcaption}
\usepackage{physics}




\def\BibTeX{{\rm B\kern-.05em{\sc i\kern-.025em b}\kern-.08em
    T\kern-.1667em\lower.7ex\hbox{E}\kern-.125emX}}


\title{In Class Tutorial, Problem 4: EE 338, Spring 2024-25}
\author{
\IEEEauthorblockN{
    \begin{tabular}{cccc}
        \begin{minipage}[t]{0.23\textwidth}
            \centering
            Anupam Rawat\\
            IIT Bombay\\
            22b3982@iitb.ac.in
        \end{minipage} & 
        \begin{minipage}[t]{0.23\textwidth}
            \centering
            Jatin Kumar\\
            IIT Bombay\\
            22b3922@iitb.ac.in
        \end{minipage} &
        \begin{minipage}[t]{0.23\textwidth}
            \centering
            Rishabh Bhardwaj\\
            IIT Bombay\\
            22b3962@iitb.ac.in
        \end{minipage} \\ 
         \\
        \\ 
    \end{tabular}
}
}

\date{March 10, 2025}


\usepackage{amsmath}
\usepackage{amssymb}
\usepackage{hyperref}
\usepackage{ulem,graphicx}
\usepackage[margin=0.5in]{geometry}

\begin{document}
\maketitle

\\

\begin{enumerate}
    \item Given $x_1[n]$ and $x_2[n]$, Prove that:- 
    \[
        DTFT(x_1[n]\cdot x_2[n]) = \frac{1}{2\pi} \int_{2\pi} X_1(e^{j\lambda}) X_2(e^{j(\omega-\lambda)})d\lambda
    \]
        
    \\
        \makebox[0pt][l]{\hspace{-7pt}\textit{Soln:}} % Aligns "Answer:" to the left
    \\

    We know that the DTFT of a discrete time signal x[n] is given by:
    \begin{equation}
        X(e^{j\omega}) = \sum_{n=-\infty}^{+\infty}x[n] e^{-j\omega n}
    \end{equation}
        
    Thus,DTFT of $x_1[n]\cdot x_2[n]$ is given by:
    \begin{equation}
        DTFT(x_1[n]\cdot x_2[n]) = \sum_{n=-\infty}^{+\infty} (x_1[n]\cdot x_2[n])e^{-j\omega n}
    \end{equation}
        
    The inverse DTFT of a signal $X(e^{j\omega})$ is given by:
    \begin{equation}
        x[n] = \frac{1}{2\pi}\int_{-\pi}^{+\pi} X(e^{j\omega})e^{j\omega n} d\omega
    \end{equation}
        
    Substituting $x_1[n]$, in eqn(2) with the formula in eqn(3):
    \begin{equation}
        DTFT(x_1[n] \cdot x_2[n]) = \sum_{n=-\infty}^{+\infty} (x_1[n]\cdot x_2[n])e^{-j\omega n} = \sum_{n=-\infty}^{+\infty} \left(\left(\frac{1}{2\pi}\int_{-\pi}^{+\pi} X_1(e^{j\lambda})e^{j\lambda n} d\lambda\right)\cdot x_2[n]\right)e^{-j\omega n}
    \end{equation}
    \begin{equation}
        DTFT(x_1[n] \cdot x_2[n]) = \frac{1}{2\pi}\int_{-\pi}^{+\pi} X_1(e^{j\lambda})  \left(\sum_{n=-\infty}^{+\infty} x_2[n] e^{j(\omega-\lambda)n} \right) d\lambda
    \end{equation}
    Using eqn(1), the inner discrete summation over n in eqn(5), can be evaluated to $X(e^{j(\omega-\lambda)})$.
    \begin{equation}
        DTFT(x_1[n] \cdot x_2[n]) = \frac{1}{2\pi}\int_{-\pi}^{+\pi} X(e^{j\lambda}) X_2(e^{j(\omega-\lambda)n}) d\lambda
    \end{equation}
    \emph{Hence Proved}
    
    
    
\end{enumerate}
\end{document}
