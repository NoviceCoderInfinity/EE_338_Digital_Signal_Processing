\documentclass{article}
\usepackage{helvet}
\usepackage{geometry}
\usepackage{graphicx}
\usepackage{amsmath}
\usepackage{hyperref}
\usepackage{xcolor}
\usepackage{titlesec}
\usepackage{microtype} % Prevents overfull hboxes by better text wrapping
\geometry{margin=1in}
\usepackage{subcaption}
\usepackage{listings}
\usepackage{color}
% Define colors for code syntax highlighting
\definecolor{codeblue}{rgb}{0.13, 0.13, 0.7}
\definecolor{codegreen}{rgb}{0, 0.5, 0}
\definecolor{codegray}{rgb}{0.5, 0.5, 0.5}
\definecolor{codepurple}{rgb}{0.58, 0, 0.82}

\lstdefinestyle{mystyle}{
    backgroundcolor=\color{white},   
    commentstyle=\color{codegreen},
    keywordstyle=\color{codeblue},
    numberstyle=\tiny\color{codegray},
    stringstyle=\color{codepurple},
    basicstyle=\ttfamily\footnotesize,
    breaklines=true,                 
    captionpos=b,                    
    keepspaces=true,                 
    numbers=left,                    
    numbersep=5pt,                  
    showspaces=false,                
    showstringspaces=false,
    showtabs=false,                  
    tabsize=2
}

\lstset{style=mystyle}

% Define colors
\definecolor{primary}{RGB}{0, 102, 204} % Blue color
\definecolor{IITBBlue}{RGB}{0, 51, 102} % IIT Bombay's signature blue

\usepackage{cite}
\usepackage{amsmath,amssymb,amsfonts}
\usepackage{algorithmic}
\usepackage{graphicx}
\usepackage{textcomp}
\usepackage{xcolor}
\usepackage{hyperref}
\usepackage{placeins}
\usepackage{graphicx}
\usepackage{subcaption}
\usepackage{physics}




\def\BibTeX{{\rm B\kern-.05em{\sc i\kern-.025em b}\kern-.08em
    T\kern-.1667em\lower.7ex\hbox{E}\kern-.125emX}}


\title{Tutorial Set 3, Problem 1: EE 338, Spring 2024-25}
\author{
\IEEEauthorblockN{
    \begin{tabular}{cccc}
        \begin{minipage}[t]{0.23\textwidth}
            \centering
            Anupam Rawat\\
            IIT Bombay\\
            22b3982@iitb.ac.in
        \end{minipage} & 
        \begin{minipage}[t]{0.23\textwidth}
            \centering
            Rishabh Bhardwaj\\
            IIT Bombay\\
            22b3962@iitb.ac.in
        \end{minipage} & 
        \begin{minipage}[t]{0.23\textwidth}
            \centering
            Jatin Kumar\\
            IIT Bombay\\
            22b3922@iitb.ac.in
        \end{minipage} \\
        \\ 
    \end{tabular}
}
}

\date{January 18, 2025}


\usepackage{amsmath}
\usepackage{amssymb}
\usepackage{hyperref}
\usepackage{ulem,graphicx}
\usepackage[margin=0.5in]{geometry}

\begin{document}
\maketitle

\\

\begin{enumerate}
    \item Let the input sequence x[n] and impulse response sequence h[n] of a discrete time LSI system be:
    \begin{enumerate}
        \item summable, i.e. \(\Sigma_n\)x[n] is finite, \(\Sigma_n\)h[n] is finite, with sums \(\Sigma_x\) and \(\Sigma_h\) respectively. Show that the output, if summable, has the sum \(\Sigma_x\)\(\Sigma_x\).
        \item absolutely summable, i.e. \(\Sigma_n|x[n]|\) is finite, \(\Sigma_n|h[n]|\) is finite, with absolute sums \(X_0\) and \(H_0\), respectively. Show that the output, if absolutely summable, has an upper absolute sum bounded by \(X_0H_0\).
    \end{enumerate}
    
    \\
        \makebox[0pt][l]{\hspace{-7pt}\textit{Soln:}} % Aligns "Answer:" to the left
    \\
    
    \begin{enumerate}
        \item 
            Both of $\Sigma_x$ and $\Sigma_h$ are finite. And we're given that:
            \[
                \Sigma_x = \Sigma_nx[n]
            \]
            \[
                \Sigma_h = \Sigma_nh[n]
            \]
            For a discrete time LSI System, with the impulse response as h[n] and the input as x[n], the output: 
            \[
                y[n] = (x \ast h)[n] = \Sigma_kx[k]h[n-k]
            \]
            Assuming y[n] is finite. The sum of the output is given as 
            \[
                \Sigma_ny[n] = \Sigma_n(\Sigma_kx[k]h[n-k])
            \]
            \[
                \Sigma_ny[n] = (...+x[0]h[0]+x[1]h[1]+...) + (...+x[1]h[0]+x[2]h[1]+...) + (...+x[2]h[0]+x[3]h[1]+...)
            \]
            \[
                \Sigma_ny[n] = ... + (...+x[0]h[0]+x[1]h[0]+x[2]h[0]+...) + (...x[1]h[1]+x[2]h[1]+x[3]h[1]+...) + ...
            \]
            \[
                \Sigma_ny[n] = ... + (...+x[0]+x[1]+x[2]+...)h[0] + (...x[1]+x[2]+x[3]+...)h[1] + ...
            \]
            \[
                \Sigma_ny[n] = (... + x[0] + x[1] + x[2] +...)(... + h[0] + h[1] + h[2] +...) = (\Sigma_nx[n])(\Sigma_nh[n])
            \]
        \item 
            We know that both $\Sigma_x$ and $\Sigma_h$ are finite. Suppose $\Sigma_n|y[n]|$ exists. \\
            \[
                \Sigma_n|y[n]| = \Sigma_n|(\Sigma_k x[k]h[n-k])|
            \]
            Now, using the property that $|\Sigma_i a_i|$ $\le$ $\Sigma_i|a_i|$:
            \[
                \Sigma_n|y[n]| \le \Sigma_n \Sigma_k |x[k] h[n-k]| = \Sigma_n \Sigma_k |x[k]|\cdot|h[n-k]|
            \]
            \[
                \Sigma_n|y[n]| \le \Sigma_k |x[k]| \cdot (\Sigma_n |h[n-k]|) = \Sigma_k |x[k]| \cdot H_0 = X_0 \cdot H_0
            \]
            Therefore, 
            \[
                \Sigma_n|y[n]| \le X_0\cdot H_0
            \]
            Thus, the output if absolutely summable, is upper bounded by $H_0\cdot X_0$
    \end{enumerate}


\end{enumerate}
\end{document}

